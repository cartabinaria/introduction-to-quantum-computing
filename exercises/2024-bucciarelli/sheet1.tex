\documentclass{article}

% Language setting
% Replace `english' with e.g. `spanish' to change the document language
\usepackage[english]{babel}

% Set page size and margins●●●●●●●●●●●●●
% Replace `letterpaper' with `a4paper' for UK/EU standard size
\usepackage[letterpaper,top=2cm,bottom=2cm,left=3cm,right=3cm,marginparwidth=1.75cm]{geometry}

% Useful packages
\usepackage{amsmath}
\usepackage{graphicx}
\usepackage[colorlinks=true, allcolors=blue]{hyperref}

\usepackage{mathtools}
\DeclarePairedDelimiter\bra{\langle}{\rvert}
\DeclarePairedDelimiter\ket{\lvert}{\rangle}
\DeclarePairedDelimiterX\braket[2]{\langle}{\rangle}{#1\,\delimsize\vert\,\mathopen{}#2}

\title{
Introduction to Quantum Computing \\  Exercise Sheet 1}

\author{Stefano Bucciarelli}

\begin{document}
\maketitle

\textbf{Exercise 1.} \\

\(z = 1 - i\) \\

I want to represent \(z\) in these forms
\begin{itemize}
    \item \(z = |z|(\cos \theta + i \sin \theta)\).
    \item \(z = |z|e^{i\theta}\)
\end{itemize} 

Calculation of \(|z|\)

\[|z| = | 1 - i | = \sqrt{1^2 + \left(-1\right)^2} = \sqrt{2} \]

Calculation of \(\theta\)

\[ 
\begin{cases} z=\sqrt{2}\left(\cos \theta + i \sin \theta\right) \\ z = 1 - i \end{cases} = 
\begin{cases} z=\sqrt{2}\cos\theta + i\sqrt{2} \sin \theta \\ z = 1 - i \end{cases} = 
\begin{cases} \sqrt{2}\cos\theta + i\sqrt{2} \sin \theta = 1 - i \\ z = 1 - i \end{cases} = \]
\[
\begin{cases} \sqrt{2}\cos\theta = 1 \\ \sqrt{2} \sin \theta = 1 \end{cases} = 
\begin{cases} \cos\theta = \frac{1}{\sqrt{2}} \\ \sin \theta = -\frac{1}{\sqrt{2}}\end{cases}  \Rightarrow
\theta = - \frac{\pi}{4}
\]

Knowing \(|z|\) and \(\theta\) I can transcribe trigonometric and exponential forms \\

So the trigonometric form is
\[z = \sqrt{2}(\cos (- \frac{\pi}{4}) + i \sin (- \frac{\pi}{4})) = \sqrt{2}(\cos  \frac{\pi}{4} - i \sin \frac{\pi}{4}) \]

And the exponential form is
\[z = \sqrt{2}e^{-i \frac{\pi}{4}}\] \\

\(z = \frac{2}{\sqrt{3} + i} = \frac{2}{\sqrt{3} + i} \frac{\sqrt{3} - i}{\sqrt{3} - i} = \frac{2(\sqrt{3} - i)}{4} = \frac{\sqrt{3}}{2} - \frac{i}{2}\) \\

I want to represent \(z\) in these forms
\begin{itemize}
    \item \(z = |z|(\cos \theta + i \sin \theta)\).
    \item \(z = |z|e^{i\theta}\)
\end{itemize} 

Calculation of \(|z|\)

\[|z| = |\frac{\sqrt{3}}{2} - \frac{i}{2}| = \sqrt{\left(\frac{\sqrt{3}}{2}\right)^2 + \left(\frac{i}{2}\right)^2} = \sqrt{\left(\frac{3}{4}\right) + \left(\frac{1}{4}\right)} = \sqrt{1} = 1 \]

Calculation of \(\theta\)

\[ 
\begin{cases} z=1(\cos \theta + i \sin \theta) \\ z = \frac{\sqrt{3}}{2} - \frac{i}{2} \end{cases} = 
\begin{cases} \cos \theta + i \sin \theta = \frac{\sqrt{3}}{2} - \frac{i}{2} \\ z = \frac{\sqrt{3}}{2} - \frac{i}{2} \end{cases} = 
\begin{cases} \cos \theta = \frac{\sqrt{3}}{2} \\ \sin \theta = - \frac{1}{2} \end{cases} \Rightarrow
\theta = - \frac{\pi}{6}
\]

So the trigonometric form is
\[z = \cos \left(- \frac{\pi}{6}\right) + i \sin \left(- \frac{\pi}{6}\right) = \cos  \frac{\pi}{6} - i \sin \frac{\pi}{6} \]

And the exponential form is
\[z = e^{-i \frac{\pi}{6}}\] \\
\pagebreak

\textbf{Exercise 2.}\\

The 5-th roots of unity are the complex numbers \(z\) which \(z^5 = 1\). I already know that these are all and only the complex numbers of the form
\(z = e^{2 \pi i \frac{k}{5}}\) for \(k = 1,...,n\). \\

So \(5roots\) which contains all 5-th roots of unity is

\[5roots = 
\{ e^{2 \pi i \frac{1}{5}}, e^{2 \pi i \frac{2}{5}}, e^{2 \pi i \frac{3}{5}}, e^{2 \pi i \frac{4}{5}}, e^{2 \pi i \frac{5}{5}}\} =
\{e^{\frac{2}{5} \pi i }, e^{ \frac{4}{5} \pi i}, e^{ \frac{6}{5} \pi i}, e^{ \frac{8}{5} \pi i}, e^{2 \pi i }\} \]

Before computing the value of elements in \(5roots\) I write some goniometric formulas that will be used later

\begin{itemize}
    \item \(\sin x = \sqrt{1 - \cos^2 x}\)
    \item \(\cos 2x = 2 \cos^2 x - 1\)
    \item \(\sin 2x = 2\sin x \ \cos x\)
    \item \(\cos \left(2 \pi - x\right) = \cos x \)
    \item \(\sin \left(2 \pi - x\right) = - \sin x \)
\end{itemize}

So let's compute the 5-th roots of unity 

\begin{enumerate}
    \item \(e^{\frac{2}{5} \pi i } = 
        \cos \left( \frac{2}{5} \pi \right) + i \sin \left( \frac{2}{5} \pi \right) = \) \\
        \( \frac{\sqrt{5} - 1}{4} + i \sqrt{1 - \cos^2 \left( \frac{2}{5} \pi \right) } = \) \\
        \( \frac{\sqrt{5} - 1}{4} + i \sqrt{1 - \left( \frac{\sqrt{5} - 1}{4}  \right)^2 } =  \) \\
        \( \frac{\sqrt{5} - 1}{4} + i \sqrt{1 - \left( \frac{6 -2 \sqrt{5} }{16}  \right) } = \) \\
        \(  \frac{\sqrt{5} - 1}{4} + i \sqrt{1 - \frac{2 \left(3 - \sqrt{5}\right) }{16} } = \) \\
       \(  \frac{\sqrt{5} - 1}{4} + i \sqrt{1 - \frac{3 - \sqrt{5}}{8} } = \) \\
        \( \frac{\sqrt{5} - 1}{4} + i \sqrt{ \frac{8 -3 + \sqrt{5}}{8} } = \) \\
        \( \frac{\sqrt{5} - 1}{4} + i \sqrt{ \frac{5 + \sqrt{5}}{8} }
        \)
    \item \(e^{\frac{4}{5} \pi i } = 
        \cos \left( \frac{4}{5} \pi \right) + i \sin \left( \frac{4}{5} \pi \right) = \) \\
        \( \cos 2\left( \frac{2}{5} \pi \right) + i \sin 2\left( \frac{2}{5} \pi \right) = \) \\
        \( 2 \cos^2\left( \frac{2}{5} \pi \right) - 1 + i 2 \sin \left( \frac{2}{5} \pi \right) \cos \left( \frac{2}{5} \pi \right) = \) \\
        \( 2 (\frac{3 - \sqrt{5}}{8}) - 1 + i 2 \sqrt{ \frac{5 + \sqrt{5}}{8}} { \frac{\sqrt{5} - 1}{4}} = = = \) \\
        \( 2 (\frac{3 - \sqrt{5}}{8}) - 1 + i 2 \sqrt{ \frac{5 + \sqrt{5}}{8}} { \frac{\sqrt{5} - 1}{4}} = = \) \\ %here do the sin simplification
        \( \frac{3 - \sqrt{5}}{4} - 1 + i 2 \sqrt{ \frac{5 + \sqrt{5}}{8}  \left(\frac{\sqrt{5} - 1}{4}\right)^2} = \) \\
        \( \frac{3 - \sqrt{5} - 4}{4} + i 2 \sqrt{ \frac{5 + \sqrt{5}}{8} \frac{3 - \sqrt{5}}{8}} = \) \\
        \( \frac{-1 - \sqrt{5}}{4} + i \sqrt{ 4 \frac{5 + \sqrt{5}}{8} \frac{3 - \sqrt{5}}{8}} = \) \\
        \(-\frac{1 + \sqrt{5}}{4}+ i \sqrt{ \frac{5 + \sqrt{5}}{8} \frac{3 - \sqrt{5}}{2}} = \) \\
        \(-\frac{1 + \sqrt{5}}{4}+ i \sqrt{ \frac{15 -5 \sqrt{5} + 3\sqrt{5} - 5}{16}}  = \) \\
        \(-\frac{1 + \sqrt{5}}{4}+ i \sqrt{ \frac{10 -2 \sqrt{5}}{16}}  = \) \\
        \(-\frac{1 + \sqrt{5}}{4}+ i \sqrt{2 \frac{5 -\sqrt{5}}{16}}  = \) \\
        \(-\frac{1 + \sqrt{5}}{4}+ i \sqrt{\frac{5 -\sqrt{5}}{8}}  \) 
    \item \(e^{\frac{6}{5} \pi i } =
        \cos \left( \frac{6}{5} \pi \right) + i \sin \left( \frac{6}{5} \pi \right) = \) \\
        \(\cos \left( 2 \pi - \frac{4}{5} \pi \right) + i \sin \left( 2 \pi - \frac{4}{5} \pi \right) = \) \\
        \(\cos \left( \frac{4}{5} \pi \right) - i \sin \left(\frac{4}{5} \pi \right) = \) \\
        \(-\frac{1 + \sqrt{5}}{4} - i \sqrt{\frac{5 -\sqrt{5}}{8}}  \) 
    \item \(e^{\frac{8}{5} \pi i } =\) 
        \(\cos \left( \frac{8}{5} \pi \right) + i \sin \left( \frac{8}{5} \pi \right) = \) \\
        \(\cos \left( 2 \pi - \frac{2}{5} \pi \right) + i \sin \left( 2 \pi - \frac{2}{5} \pi \right) = \) \\
        \(\cos \left( \frac{2}{5} \pi \right) - i \sin \left(\frac{2}{5} \pi \right) = \) \\
        \( \frac{\sqrt{5} - 1}{4} - i \sqrt{ \frac{5 + \sqrt{5}}{8} } \)
    \item \(e^{2 \pi i } =\) 
        \(\cos \left( 2 \pi \right) + i \sin \left( 2 \pi \right) = 1\) \\
\end{enumerate}
\pagebreak
\\
\textbf{Exercise 3.}\\

\(
\ket{101} = 
\ket{1} \otimes \ket{0} \otimes \ket{1} = 
\ket{1} \otimes \left(\ket{0} \otimes \ket{1}\right) =  
\ket{1} \otimes \begin{pmatrix} 1 \begin{pmatrix} 0 \\ 1 \end{pmatrix} \\ 0  \begin{pmatrix} 0 \\ 1 \end{pmatrix} \end{pmatrix} =
\ket{1} \otimes \begin{pmatrix} 0 \\ 1 \\ 0 \\ 0 \end{pmatrix} =
\begin{pmatrix} 0 \begin{pmatrix} 0 \\ 1 \\ 0 \\ 0 \end{pmatrix} \\ 1 \begin{pmatrix} 0 \\ 1 \\ 0 \\ 0 \end{pmatrix} \end{pmatrix} =
\begin{pmatrix} 0 \\ 0 \\ 0 \\ 0 \\ 0 \\ 1 \\ 0 \\ 0 \end{pmatrix}
\)

\(
\ket{01} \otimes \ket{0} = 
\left( \ket{0} \otimes \ket{1} \right) \otimes \ket{0} = 
\begin{pmatrix} 0 \\ 1 \\ 0 \\ 0 \end{pmatrix} \otimes \ket{0} =
\begin{pmatrix} 0 \begin{pmatrix} 1 \\ 0 \end{pmatrix} \\ 1 \begin{pmatrix} 1 \\ 0 \end{pmatrix} \\ 0 \begin{pmatrix} 1 \\ 0 \end{pmatrix} \\ 0 \begin{pmatrix} 1 \\ 0 \end{pmatrix} \end{pmatrix} =
\begin{pmatrix} 0 \\ 0 \\ 1 \\ 0 \\ 0 \\ 0 \\ 0 \\ 0 \end{pmatrix} = 
\ket{010}
\)

\(
\ket{111} = 
\ket{1} \otimes \ket{1} \otimes \ket{1} = 
\ket{1} \otimes \left(\ket{1} \otimes \ket{1}\right) =  
\ket{1} \otimes \begin{pmatrix} 0 \begin{pmatrix} 0 \\ 1 \end{pmatrix} \\ 1  \begin{pmatrix} 0 \\ 1 \end{pmatrix} \end{pmatrix} =
\ket{1} \otimes \begin{pmatrix} 0 \\ 0 \\ 0 \\ 1 \end{pmatrix} =
\begin{pmatrix} 0 \begin{pmatrix} 0 \\ 0 \\ 0 \\ 1 \end{pmatrix} \\ 1 \begin{pmatrix} 0 \\ 0 \\ 0 \\ 1 \end{pmatrix} \end{pmatrix} =
\begin{pmatrix} 0 \\ 0 \\ 0 \\ 0 \\ 0 \\ 0 \\ 0 \\ 1 \end{pmatrix}
\)

\(
\frac{1}{\sqrt{3}} \left(\ket{101} + \ket{010} + \ket{111} \right) = \frac{1}{\sqrt{3}} \begin{pmatrix} 0 \\ 0 \\ 0 \\ 0 \\ 0 \\ 1 \\ 0 \\ 0 \end{pmatrix} + \begin{pmatrix} 0 \\ 0 \\ 1 \\ 0 \\ 0 \\ 0 \\ 0 \\ 0 \end{pmatrix} +  \begin{pmatrix} 0 \\ 0 \\ 0 \\ 0 \\ 0 \\ 0 \\ 0 \\ 1 \end{pmatrix} = 
\frac{1}{\sqrt{3}} \begin{pmatrix} 0 \\ 0 \\ 1 \\ 0 \\ 0 \\ 1 \\ 0 \\ 1 \end{pmatrix} =
\begin{pmatrix} 0 \\ 0 \\ \frac{1}{\sqrt{3}}  \\ 0 \\ 0 \\ \frac{1}{\sqrt{3}}  \\ 0 \\ \frac{1}{\sqrt{3}}  \end{pmatrix}
\)
\pagebreak
\\
\textbf{Exercise 4.}\\

To check if these vector are entangled I try to decompose every vector in the form

\(
\left( \alpha_{0} \ket{0} + \alpha_{1} \ket{1} \right) \otimes \left( \beta_{0} \ket{0} + \beta_{1} \ket{1} \right) \otimes \left( \gamma_{0} \ket{0} + \gamma_{1} \ket{1} \right) = \\
\alpha_{0} \beta_{0} \gamma_{0} \ket{000} + 
\alpha_{0} \beta_{0} \gamma_{1} \ket{001} + 
\alpha_{0} \beta_{1} \gamma_{0} \ket{010} + 
\alpha_{0} \beta_{1} \gamma_{1} \ket{011} + 
\alpha_{1} \beta_{0} \gamma_{0} \ket{100} + 
\alpha_{1} \beta_{0} \gamma_{1} \ket{101} +
\alpha_{1} \beta_{1} \gamma_{0} \ket{110} +
\alpha_{1} \beta_{1} \gamma_{1} \ket{111}
\)

\[
\frac{1}{\sqrt{3}} \left(\ket{001} + \ket{111} + \ket{011} \right)
\]

\(
\frac{1}{\sqrt{3}} \left(\ket{001} + \ket{111} + \ket{011} \right) \Rightarrow \begin{cases} 
\alpha_{0} \beta_{0} \gamma_{0} = 0 \\
\alpha_{0} \beta_{0} \gamma_{1} = \frac{1}{\sqrt{3}} \\
\alpha_{0} \beta_{1} \gamma_{0} = 0 \\
\alpha_{0} \beta_{1} \gamma_{1} = \frac{1}{\sqrt{3}} \\
\alpha_{1} \beta_{0} \gamma_{0} = 0 \\
\alpha_{1} \beta_{0} \gamma_{1} = 0 \\
\alpha_{1} \beta_{1} \gamma_{0} = 0 \\
\alpha_{1} \beta_{1} \gamma_{1} = \frac{1}{\sqrt{3}} 
\end{cases}
\Rightarrow
\begin{cases} 
 \gamma_{0} = 0 \\
 \alpha_{0} \beta_{0} \neq 0 \\ 
 \alpha_{0} \beta_{1} \neq 0 \\ 
 \alpha_{1} \beta_{1} \neq 0 \\ 
 \alpha_{1} \beta_{0} = 0 \Rightarrow \alpha_{1} = 0 ~or~ \beta_{0} = 0 \Rightarrow \alpha_{1} \beta_{1} = 0 ~or~ \alpha_{0} \beta_{0} = 0
\end{cases}
\) 
\\
\\
But it is impossible to have \( \alpha_{1} \beta_{1} \neq 0\), \(\alpha_{0} \beta_{0} \neq 0 \) and \(\alpha_{1} \beta_{1} = 0 ~or~ \alpha_{0} \beta_{0} = 0\) so it is impossible to decompose the original the vector, so the vector is entangled
\\
\\
\[
\frac{1}{\sqrt{2}} \left(\ket{011} + \ket{100}\right)
\] 

\(
\frac{1}{\sqrt{2}} \left(\ket{011} + \ket{100}\right) \Rightarrow \begin{cases} 
\alpha_{0} \beta_{0} \gamma_{0} = 0 \\
\alpha_{0} \beta_{0} \gamma_{1}  = 0 \\
\alpha_{0} \beta_{1} \gamma_{0}  = 0 \\
\alpha_{0} \beta_{1} \gamma_{1}  = \frac{1}{\sqrt{2}} \\
\alpha_{1} \beta_{0} \gamma_{0}  = \frac{1}{\sqrt{2}} \\
\alpha_{1} \beta_{0} \gamma_{1}  = 0 \\
\alpha_{1} \beta_{1} \gamma_{0}  = 0 \\
\alpha_{1} \beta_{1} \gamma_{1} = 0
\end{cases}
\Rightarrow
\begin{cases} 
\alpha_{0} \beta_{0} \gamma_{0} = 0 \\
\alpha_{0} \beta_{1} \gamma_{1} = \frac{1}{\sqrt{2}} \\
\alpha_{1} \beta_{0} \gamma_{0} = \frac{1}{\sqrt{2}} \\
\end{cases}
\Rightarrow
\begin{cases} 
\alpha_{0} = 0 ~or~ \beta_{0} = 0 ~or~ \gamma_{0} = 0 \\
\alpha_{0} \neq 0 ~and~ \beta_{1} \neq 0 ~and~ \gamma_{1} \neq 0 \\
\alpha_{1} \neq 0 ~and~ \beta_{0} \neq 0 ~and~ \gamma_{0} \neq 0 \\
\end{cases}
\Rightarrow
\begin{cases} 
\alpha_{0} = 0 ~or~ \beta_{0} = 0 ~or~ \gamma_{0} = 0 \\
\alpha_{0} \neq 0 ~and~ \beta_{1} \neq 0 ~and~ \gamma_{0} \neq 0 \\
\end{cases}
\)
\\
\\

But it is impossible to have \(\alpha_{0} = 0 ~or~ \beta_{0} = 0 ~or~ \gamma_{0} = 0\) and \(\alpha_{0} \neq 0 ~and~ \beta_{1} \neq 0 ~and~ \gamma_{0} \neq = 0\) so it is impossible to decompose the original the vector, so the vector is entangled
\\
\\
\[
\frac{1}{\sqrt{2}} \left(\ket{011} + \ket{101}\right)
\] 

\(
\frac{1}{\sqrt{2}} \left(\ket{011} + \ket{101}\right) \Rightarrow \begin{cases} 
\alpha_{0} \beta_{0} \gamma_{0} = 0 \\
\alpha_{0} \beta_{0} \gamma_{1}  = 0 \\
\alpha_{0} \beta_{1} \gamma_{0}  = 0 \\
\alpha_{0} \beta_{1} \gamma_{1}  = \frac{1}{\sqrt{2}} \\
\alpha_{1} \beta_{0} \gamma_{1}  = 0 \\
\alpha_{1} \beta_{0} \gamma_{1}  = \frac{1}{\sqrt{2}} \\
\alpha_{1} \beta_{1} \gamma_{0}  = 0 \\
\alpha_{1} \beta_{1} \gamma_{1} = 0
\end{cases}
\Rightarrow
\begin{cases} 
\gamma_{0} = 0 \\
\alpha_{0} \beta_{1} \gamma_{1} \neq 0 \\
\alpha_{1} \beta_{0} \gamma_{1} \neq 0 \\
\alpha_{0} \beta_{0} \gamma_{1} = 0 \\
\alpha_{1} \beta_{1} \gamma_{1} = 0 \\
\end{cases}
\Rightarrow
\begin{cases} 
\alpha_{0} \beta_{1} \neq 0 \\
\alpha_{1} \beta_{0} \neq 0 \\
\alpha_{0} = 0 ~or~ \beta_{0} = 0 \\
\alpha_{1} = 0 ~or~ \beta_{1} = 0 \\
\end{cases}
\) 
\\
\\
There is no solution, because if \(\alpha_{0} = 0 ~or~ \beta_{0} = 0\) and \(\alpha_{1} = 0 ~or~ \beta_{1} = 0\) and \(\alpha_{0} \beta_{1} \neq 0 \) imply that \(\alpha_{0} \beta_{1} = 0\) but instead there is \(\alpha_{1} \beta_{0} \neq 0 \)
\\
But it is impossible to have \(\alpha_{0} = 0 ~or~ \beta_{0} = 0 ~or~ \gamma_{0} = 0\) and \(\alpha_{0} \neq 0 ~and~ \beta_{1} \neq 0 ~and~ \gamma_{0} \neq 0\) so it is impossible to decompose the original the vector, so the vector is entangled
\pagebreak
\\
\textbf{Exercise 5.}
\\\\
\(
\frac{1}{\sqrt{6}} \left(\ket{01}\bra{00} + 5 \ket{01}\bra{01} + 3 \ket{10}\bra{11} + 2i \ket{10}\bra{11}\right) \frac{1}{\sqrt{2}} \left(\ket{00} + \ket{11} \right) = \\\\
\frac{1}{\sqrt{12}} 
((\ket{01}\bra{00})\ket{00} + 5 (\ket{01}\bra{01})\ket{00} + 3 (\ket{10}\bra{11})\ket{00} + 2i (\ket{10}\bra{11})\ket{00} + (\ket{01}\bra{00})\ket{11} + 5 (\ket{01}\bra{01})\ket{11} + 3 (\ket{10}\bra{11})\ket{11} + 2i (\ket{10}\bra{11})\ket{11}) = \\\\
\frac{1}{\sqrt{12}} ((\bra{00}\ket{00})\ket{01} + 5 (\bra{01}\ket{00})\ket{01} + 3 (\bra{11}\ket{00})\ket{10} + 2i (\bra{11}\ket{00})\ket{10} + (\bra{00}\ket{11})\ket{01} + 5 (\bra{01}\ket{11})\ket{01} + 3 (\bra{11}\ket{11})\ket{10} + 2i (\bra{11}\ket{11})\ket{10}) = \\\\
\frac{1}{\sqrt{12}} (1 \ket{01} + 5 \cdot 0\ket{01} + 3 \cdot 0\ket{10} + 2i \cdot 0\ket{10} + 0\ket{01} + 5 \cdot 0\ket{01} + 3 \cdot 1\ket{10} + 2i \cdot 1\ket{10}) = \\\\
\frac{1}{\sqrt{12}} (\ket{01} + 3\ket{10} + 2i \ket{10}) = \\\\
\frac{1}{\sqrt{12}} \ket{01} + \frac{3 + 2i}{\sqrt{12}} \ket{10}
\)
\pagebreak
\\
\textbf{Exercise 6.}\\

\[
M= \left( 
    \begin{array}{cc} 
        1 & -1 \\
        1 & 1
    \end{array} 
\right)
\]
\\
For finding eigenvalues I find \(\lambda\) that resolve the equation \(det\left(\lambda I - M \right) = 0\)

\[\lambda I - M = 
\left( 
    \begin{array}{cc} 
       \lambda - 1 & 1 \\
        -1 & \lambda - 1
    \end{array} 
\right)\]
\\

\[
det\left(\lambda I - M \right) = 0 \Rightarrow
(\lambda -1)(\lambda -1) + 1 = 0 \Rightarrow
(\lambda - 1)^2 = -1 \Rightarrow
\lambda - 1 = \pm i 
\]
\\
So \(\lambda_{1} = 1 + i\), \(\lambda_{2} = 1 - i\)
\\
\\
Considering eigenvalue \(\lambda_{1} = 1 + i\)

\[\left(\lambda_1 I - M \right)\begin{pmatrix} x \\ y \end{pmatrix} = \begin{pmatrix} 0 \\ 0 \end{pmatrix} \]
\[
\left( 
    \begin{array}{cc} 
       i & 1 \\
       -1 & i
    \end{array} 
\right)\begin{pmatrix} x \\ y \end{pmatrix} = \begin{pmatrix} 0 \\ 0 \end{pmatrix} \Rightarrow 
\begin{cases} 
y + ix =0 \\ 
-x + iy = 0
\end{cases} =
\begin{cases} 
y = -ix \\ 
x = iy 
\end{cases} 
\]
\\
\\
Orthonormal solutions must further satisfy \(|x|^2 + |y|^2 = 1\), possible solutions are \(x = \frac{1}{\sqrt{2}}, y = -\frac{i}{\sqrt{2}}\), so I can deduce that the eigenvector \(\ket{\phi_1} = \frac{1}{\sqrt{2}}\begin{pmatrix} 1 \\ -i \end{pmatrix} = \frac{1}{\sqrt{2}}(\ket{0} - i \ket{1}) \), now I prove that
\(M \ket{\phi_1} = \lambda_1 \ket{\phi_1}\)

\[ M \ket{\phi_1} =  \left( 
    \begin{array}{cc} 
        1 & -1 \\
        1 & 1
    \end{array} 
\right)\frac{1}{\sqrt{2}}\begin{pmatrix} 1 \\ -i \end{pmatrix} = 
\frac{1}{\sqrt{2}}\begin{pmatrix} 1 + i \\  1 - i \end{pmatrix} = 
(1 + i) \frac{1}{\sqrt{2}}\begin{pmatrix} 1 \\  - i \end{pmatrix} =
\lambda_1 \ket{\phi_1}
\]
\\
\\
\\
Considering eigenvalue \(\lambda_2 = 1 - i\)

\[\left(\lambda_2 I - M \right)\begin{pmatrix} x \\ y \end{pmatrix} = \begin{pmatrix} 0 \\ 0 \end{pmatrix} \]
\[
\left( 
    \begin{array}{cc} 
       -i & 1 \\
       -1 & -i
    \end{array} 
\right)\begin{pmatrix} x \\ y \end{pmatrix} = \begin{pmatrix} 0 \\ 0 \end{pmatrix} \Rightarrow 
\begin{cases} 
y - ix =0 \\ 
-x - iy = 0
\end{cases} =
\begin{cases} 
y = ix \\ 
x = -iy 
\end{cases} 
\]
\\
\\
Orthonormal solutions must further satisfy \(|x|^2 + |y|^2 = 1\), possible solutions are \(x = -\frac{i}{\sqrt{2}}, y = \frac{1}{\sqrt{2}}\), so I can deduce that the eigenvector \(\ket{\phi_2} = \frac{1}{\sqrt{2}}\begin{pmatrix} -i \\ 1 \end{pmatrix} = \frac{1}{\sqrt{2}}(- 1 \ket{0} + \ket{1}) \), now I prove that
\(M \ket{\phi_2} = \lambda_2 \ket{\phi_2}\)

\[ M \ket{\phi_2} =  \left( 
    \begin{array}{cc} 
        1 & -1 \\
        1 & 1
    \end{array} 
\right)\frac{1}{\sqrt{2}}\begin{pmatrix} -i \\ 1 \end{pmatrix} = 
\frac{1}{\sqrt{2}}\begin{pmatrix} -1 - i \\  1 - i \end{pmatrix} = 
(1 - i) \frac{1}{\sqrt{2}}\begin{pmatrix} -i \\  1 \end{pmatrix} =
\lambda_2 \ket{\phi_2}
\]

\end{document}