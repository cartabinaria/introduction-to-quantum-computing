\documentclass{article}

% Language setting
% Replace `english' with e.g. `spanish' to change the document language
\usepackage[english]{babel}

% Set page size and margins
% Replace `letterpaper' with `a4paper' for UK/EU standard size
\usepackage[letterpaper,top=2cm,bottom=2cm,left=3cm,right=3cm,marginparwidth=1.75cm]{geometry}




% Useful packages
\usepackage{amssymb}
\usepackage{amsmath}
\usepackage{graphicx}
\usepackage[colorlinks=true, allcolors=blue]{hyperref}

\usepackage{mathtools}
\DeclarePairedDelimiter\bra{\langle}{\rvert}
\DeclarePairedDelimiter\ket{\lvert}{\rangle}
\DeclarePairedDelimiterX\braket[2]{\langle}{\rangle}{#1\,\delimsize\vert\,\mathopen{}#2}

\usepackage{tikz}
\usetikzlibrary{quantikz2}


\title{
Introduction to Quantum Computing \\  Exercise Sheet 2}

\author{Stefano Bucciarelli}

\begin{document}
\maketitle

\textbf{Exercise 1.} \\

As suggested, I check the behavior of the two circuits over the basic states: \\
\begin{quantikz}[slice all, slice
titles=$\lvert{\psi_{\col}^{ij}}\rangle$,slice style=red,slice label
style={}]
\ket{i} & \gate{H} & \ctrl{1} & \gate{H} & \\
\ket{j} & \gate{H} & \gate{X} & \gate{H} & 
\end{quantikz}

\begin{quantikz}[slice all, slice
titles=$\lvert{\psi_{4}^{ij}}\rangle$,slice style=red,slice label
style={}]
\ket{i} & \gate{X} & \\
\ket{j} & \ctrl{-1} &
\end{quantikz}

I first compute the $CNOT$ on Hadamard's basis, which is helpful then.
\[CNOT \ket{+}\ket{+}  =  \left(
    \begin{array}{cccc} 
        1 & 0 & 0 & 0\\
        0 & 1 & 0 & 0\\
        0 & 0 & 0 & 1\\
        0 & 0 & 1 & 0
    \end{array} :
    
\right)\begin{pmatrix} \frac{1}{\sqrt{2}} \begin{pmatrix} \frac{1}{\sqrt{2}} \\ \frac{1}{\sqrt{2}} \end{pmatrix} \\ \frac{1}{\sqrt{2}} \begin{pmatrix} \frac{1}{\sqrt{2}} \\ \frac{1}{\sqrt{2}} \end{pmatrix} \end{pmatrix} = \left(
    \begin{array}{cccc} 
        1 & 0 & 0 & 0\\
        0 & 1 & 0 & 0\\
        0 & 0 & 0 & 1\\
        0 & 0 & 1 & 0
    \end{array} 
\right)\begin{pmatrix} \frac{1}{2} \\ \frac{1}{2} \\ \frac{1}{2} \\ \frac{1}{2}  \end{pmatrix} = \begin{pmatrix} \frac{1}{2} \\ \frac{1}{2} \\ \frac{1}{2} \\ \frac{1}{2}  \end{pmatrix} = \ket{+}\ket{+} \]


\[CNOT \ket{+}\ket{-}  =  \left(
    \begin{array}{cccc} 
        1 & 0 & 0 & 0\\
        0 & 1 & 0 & 0\\
        0 & 0 & 0 & 1\\
        0 & 0 & 1 & 0
    \end{array} 
\right)\begin{pmatrix} \frac{1}{\sqrt{2}} \begin{pmatrix} \frac{1}{\sqrt{2}} \\ -\frac{1}{\sqrt{2}} \end{pmatrix} \\ \frac{1}{\sqrt{2}} \begin{pmatrix} \frac{1}{\sqrt{2}} \\ -\frac{1}{\sqrt{2}} \end{pmatrix} \end{pmatrix} = \left(
    \begin{array}{cccc} 
        1 & 0 & 0 & 0\\
        0 & 1 & 0 & 0\\
        0 & 0 & 0 & 1\\
        0 & 0 & 1 & 0
    \end{array} 
\right)\begin{pmatrix} \frac{1}{2} \\ -\frac{1}{2} \\ \frac{1}{2} \\ -\frac{1}{2}  \end{pmatrix} = \begin{pmatrix} \frac{1}{2} \\ -\frac{1}{2} \\ -\frac{1}{2} \\ \frac{1}{2}  \end{pmatrix} = \ket{-}\ket{-} \]


\[CNOT \ket{-}\ket{+}  =  \left(
    \begin{array}{cccc} 
        1 & 0 & 0 & 0\\
        0 & 1 & 0 & 0\\
        0 & 0 & 0 & 1\\
        0 & 0 & 1 & 0
    \end{array} 
\right)\begin{pmatrix} \frac{1}{\sqrt{2}} \begin{pmatrix} \frac{1}{\sqrt{2}} \\ \frac{1}{\sqrt{2}} \end{pmatrix} \\ - \frac{1}{\sqrt{2}} \begin{pmatrix} \frac{1}{\sqrt{2}} \\ \frac{1}{\sqrt{2}} \end{pmatrix} \end{pmatrix} = \left(
    \begin{array}{cccc} 
        1 & 0 & 0 & 0\\
        0 & 1 & 0 & 0\\
        0 & 0 & 0 & 1\\
        0 & 0 & 1 & 0
    \end{array} 
\right)\begin{pmatrix} \frac{1}{2} \\ \frac{1}{2} \\ -\frac{1}{2} \\ -\frac{1}{2}  \end{pmatrix} = \begin{pmatrix} \frac{1}{2} \\ \frac{1}{2} \\ -\frac{1}{2} \\ -\frac{1}{2}  \end{pmatrix} = \ket{-}\ket{+}\]


\[CNOT \ket{-}\ket{-}  =  \left(
    \begin{array}{cccc} 
        1 & 0 & 0 & 0\\
        0 & 1 & 0 & 0\\
        0 & 0 & 0 & 1\\
        0 & 0 & 1 & 0
    \end{array} 
\right)\begin{pmatrix} \frac{1}{\sqrt{2}} \begin{pmatrix} \frac{1}{\sqrt{2}} \\ -\frac{1}{\sqrt{2}} \end{pmatrix} \\ - \frac{1}{\sqrt{2}} \begin{pmatrix} \frac{1}{\sqrt{2}} \\ -\frac{1}{\sqrt{2}} \end{pmatrix} \end{pmatrix} = \left(
    \begin{array}{cccc} 
        1 & 0 & 0 & 0\\
        0 & 1 & 0 & 0\\
        0 & 0 & 0 & 1\\
        0 & 0 & 1 & 0
    \end{array} 
\right)\begin{pmatrix} \frac{1}{2} \\ -\frac{1}{2} \\ -\frac{1}{2} \\ \frac{1}{2}  \end{pmatrix} = \begin{pmatrix} \frac{1}{2} \\ -\frac{1}{2} \\ \frac{1}{2} \\ -\frac{1}{2}  \end{pmatrix} = \ket{+}\ket{-}\]
Now for $i,j \in [0,1]$ I compute $\psi_{1}^{ij}, \psi_{2}^{ij}, \psi_{3}^{ij}, \psi_{4}^{ij}$ and compare $\psi_{3}^{ij}$ with $\psi_{4}^{ij}$

\[\psi_{1}^{00} = H\ket{0} \otimes H\ket{0} = \ket{+}\ket{+}\]
\[\psi_{2}^{00} = CNOT \ket{+}\ket{+} = \ket{+}\ket{+}\]
\[\psi_{3}^{00} = H\ket{+}H\ket{+} = \ket{0}\ket{0}\]
\[\psi_{4}^{00} = \ket{0}\ket{0} \] \\
 In $\psi_{4}^{00}$ the first qubit is unchanged because the control (second) qubit is $\ket{0}$. So $\psi_{3}^{00}$ is equal to $\psi_{4}^{00}$.
 \\
\[\psi_{1}^{01} = H\ket{0} \otimes H\ket{1} = \ket{+}\ket{-}\]
\[\psi_{2}^{01} = CNOT \ket{+}\ket{-} = \ket{-}\ket{-}\]
\[\psi_{3}^{01} = H\ket{-}H\ket{-} = \ket{1}\ket{1}\]
\\
\[\psi_{4}^{01} = (NOT\ket{0})\ket{1} = \ket{1}\ket{1} \] In $\psi_{4}^{01}$ the first qubit is changed because the control (second) qubit is $\ket{1}$. So $\psi_{3}^{01}$ is equal to $\psi_{4}^{01}$.
\\
\[\psi_{1}^{10} = H\ket{1} \otimes H\ket{0} = \ket{-}\ket{+}\]
\[\psi_{2}^{10} = CNOT \ket{-}\ket{-+} = \ket{-}\ket{+}\]
\[\psi_{3}^{10} = H\ket{-}H\ket{+} = \ket{1}\ket{0}\]

\[\psi_{4}^{10} = \ket{1}\ket{0} \] In $\psi_{4}^{10}$ the first qubit is unchanged because the control (second) qubit is $\ket{0}$. So $\psi_{3}^{10}$ is equal to $\psi_{4}^{10}$.
\\

\[\psi_{1}^{11} = H\ket{1} \otimes H\ket{1} = \ket{-}\ket{-}\]
\[\psi_{2}^{11} = CNOT \ket{-}\ket{-} = \ket{+}\ket{-}\]
\[\psi_{3}^{11} = H\ket{+}H\ket{-} = \ket{0}\ket{1}\]

\[\psi_{4}^{11} = (NOT\ket{1})\ket{1} = \ket{0}\ket{1} \] In $\psi_{4}^{11}$ the first qubit is changed because the control (second) qubit is $\ket{1}$. So $\psi_{3}^{11}$ is equal to $\psi_{4}^{11}$.\\

The process is the same for the other circuit.\\

\begin{quantikz}[slice all, slice
titles=$\lvert{\psi_{\col}^{ij}}\rangle$,slice style=red,slice label
style={}]
\ket{i} & \gate{H} & \ctrl{1} & & \gate{H} & \\
\ket{j} & \gate{H} & \gate{X} & \gate{X} & \gate{H} & 
\end{quantikz}

\begin{quantikz}[slice
titles=$\lvert{\psi_{\col}^{ij}}\rangle$,slice style=blue,slice label
style={}]
\ket{i} & \gate{X} \slice{$\lvert{\psi_{5}^{ij}}\rangle$}  & \slice{$\lvert{\psi_{6}^{ij}}\rangle$} & \\
\ket{j} & \ctrl{-1} & \gate{Z} & 
\end{quantikz}

For $i,j \in [0,1]$ I compute $\psi_{1}^{ij}, \psi_{2}^{ij}, \psi_{3}^{ij}, \psi_{4}^{ij}, \psi_{5}^{ij}, \psi_{6}^{ij}, $ and compare $\psi_{4}^{ij}$ with $\psi_{6}^{ij}$, recalling that $NOT$ has eigenvector $\ket{+}$ and $\ket{-}$ with respective eigenvalues $1$ and $-1$, so $NOT\ket{+} = \ket{+}$ and $NOT\ket{-} = -\ket{-}$

\[\psi_{1}^{00} = H\ket{0} \otimes H\ket{0} = \ket{+}\ket{+}\]
\[\psi_{2}^{00} = CNOT \ket{+}\ket{+} = \ket{+}\ket{+}\]
\[\psi_{3}^{00} = \ket{+}NOT\ket{+} = \ket{+}\ket{+}\]
\[\psi_{4}^{00} = H\ket{+}H\ket{+} = \ket{0}\ket{0}\]
\\
\[\psi_{5}^{00} = \ket{0}\ket{0} \]
\[\psi_{6}^{00} = \ket{0}(Z\ket{0}) = \ket{0}\ket{0} \]
So $\psi_{4}^{00}$ is equal to $\psi_{6}^{00}$\\

\[\psi_{1}^{01} = H\ket{0} \otimes H\ket{1} = \ket{+}\ket{-}\]
\[\psi_{2}^{01} = CNOT \ket{+}\ket{-} = \ket{-}\ket{-}\]
\[\psi_{3}^{01} = \ket{-}NOT\ket{-} = \ket{-}(-\ket{-})\]
\[\psi_{4}^{01} = H\ket{-}H(-\ket{-}) = \ket{1}(-\ket{1})\]
\\
\[\psi_{5}^{01} = \ket{0}\ket{1} \]
\[\psi_{6}^{01} = \ket{1}(Z\ket{1}) = \ket{1}(-\ket{1}) \]
So $\psi_{4}^{01}$ is equal to $\psi_{6}^{01}$\\

\[\psi_{1}^{10} = H\ket{1} \otimes H\ket{0} = \ket{-}\ket{+}\]
\[\psi_{2}^{10} = CNOT \ket{-}\ket{+} = \ket{-}\ket{+}\]
\[\psi_{3}^{10} = \ket{-}NOT\ket{+} = \ket{-}\ket{+}\]
\[\psi_{4}^{10} = H\ket{-}H\ket{+} = \ket{1}\ket{0}\]
\\
\[\psi_{5}^{10} = \ket{1}\ket{0} \]
\[\psi_{6}^{10} = \ket{1}(Z\ket{0}) = \ket{1}\ket{0} \]
So $\psi_{4}^{10}$ is equal to $\psi_{6}^{10}$\\

\[\psi_{1}^{11} = H\ket{1} \otimes H\ket{1} = \ket{-}\ket{-}\]
\[\psi_{2}^{11} = CNOT \ket{-}\ket{-} = \ket{+}\ket{-}\]
\[\psi_{3}^{11} = \ket{+}NOT\ket{-} = \ket{+}(-\ket{-})\]
\[\psi_{4}^{11} = H\ket{+}H(-\ket{-}) = \ket{0}(-\ket{1})\]
\\
\[\psi_{5}^{11} = \ket{1}\ket{1} \]
\[\psi_{6}^{11} = \ket{0}(Z\ket{1}) = \ket{0}(-\ket{1}) \]
So $\psi_{4}^{11}$ is equal to $\psi_{6}^{11}$\\

I now demonstrate that, if two quantum circuits have the same behavior over the basis states, then they will have the same behavior for each qubit in input.

Let's consider these quantum circuits: \\

\begin{quantikz}
\ket{q_{0}} & \gate[2]{U} & \\
\ket{q_{1}} &&
\end{quantikz}
\begin{quantikz}
\ket{q_{0}} & \gate[2]{U'} & \\
\ket{q_{1}} &&
\end{quantikz}

That have the same behaviour over basis states where $\ket{q_0} = \alpha_0 \ket{0} + \beta_0 \ket{1}$ and $\ket{q_1} = \alpha_1 \ket{0} + \beta_1 \ket{1}$

So the first circuit perform 
\[U\ket{q_0}\ket{q_1} = U [(\alpha_0 \ket{0} + \beta_0 \ket{1})\otimes(\alpha_1 \ket{0} + \beta_1 \ket{1})] = U (\alpha_0\alpha_1 \ket{0}\ket{0} + \alpha_0\beta_1 \ket{0}\ket{1} + \alpha_1\beta_0 \ket{1}\ket{0} + \alpha_1\beta_1 \ket{1}\ket{1}) = \]
\[\alpha_0\alpha_1 U\ket{0}\ket{0} + \alpha_0\beta_1 U\ket{0}\ket{1} + \alpha_1\beta_0 U\ket{1}\ket{0} + \alpha_1\beta_1 U\ket{1}\ket{1} = \alpha_0\alpha_1 U'\ket{0}\ket{0} + \alpha_0\beta_1 U'\ket{0}\ket{1} + \alpha_1\beta_0 U'\ket{1}\ket{0} + \alpha_1\beta_1 U'\ket{1}\ket{1} = \]
\[U' (\alpha_0\alpha_1 \ket{0}\ket{0} + \alpha_0\beta_1 \ket{0}\ket{1} + \alpha_1\beta_0 \ket{1}\ket{0} + \alpha_1\beta_1 \ket{1}\ket{1}) = U' [(\alpha_0 \ket{0} + \beta_0 \ket{1})\otimes(\alpha_1 \ket{0} + \beta_1 \ket{1})] = U'\ket{q_0}\ket{q_1}\]

The two generic circuits are equals, so the initial circuits compute the same operation.

\pagebreak

\textbf{Exercise 2.} \\

In the Deutsch circuit of a f function, that return $\ket{1}$ on the first qubit if and only if $f(0) = f(1)$:

\begin{quantikz}
\ket{0} & \gate{H} & \gate[2]{U_f} & \gate{H} & \\
\ket{-} &&&&
\end{quantikz}

Since we need to perform a parity measure, we must change the second qubit in order to entangle it with the first.\\

So the solution proposed is 

\begin{quantikz}
\ket{0} & \gate{H} & \gate[2]{U_f} \slice{$\lvert{\psi_1}\rangle$} & \ctrl{1} \slice{$\lvert{\psi_2}\rangle$}  &\gate{H} & \slice{$\lvert{\psi_3}\rangle$} & \\
\ket{-} &&&\gate{X}&\gate{H} &&
\end{quantikz}\\

So considering $f$ as constant ($f(0) = f(1)$) it is \\
$\ket{\psi_1} = \ket{+}\ket{-}$ simply is the same respect Deuscht algorithm \\
$\ket{\psi_2} = CNOT\ket{+}\ket{-} = \ket{-}\ket{-} $ as seen previously\\
$\ket{\psi_3} = H\ket{-}H\ket{-} = \ket{1}\ket{1}$ \\ so if f is constant the parity measurement yield 0 with probability 1.\\

If f is not constant ($f(0) \neq f(1)$) it is \\
$\ket{\psi_1} = \ket{-}\ket{-}$ simply is the same respect Deuscht algorithm \\
$\ket{\psi_2} = CNOT\ket{-}\ket{-} = \ket{+}\ket{-} $ as seen previously\\
$\ket{\psi_3} = H\ket{+}H\ket{-} = \ket{1}\ket{0}$ \\ so if f is not constant the parity measurement yield 1 with probability 1.
ck
\end{document}